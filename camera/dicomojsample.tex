\documentclass[Japanese]{dicomopapers}

%\documentclass[Japanese,noauthor]{dicomopapers}

% \usepackage[dvips]{graphicx}
\usepackage{latexsym}
\usepackage[dvipdfmx]{graphicx}

\usepackage[utf8]{inputenc}
\usepackage{array}
\usepackage{amssymb} % 数学記号用
\usepackage{scalefnt}
\usepackage{makecell}
\usepackage{float}
\usepackage{booktabs} 
\usepackage{listings}
\usepackage{amsmath}

\providecommand{\newblock}{}


\def\Underline{\setbox0\hbox\bgroup\let\\\endUnderline}
\def\endUnderline{\vphantom{y}\egroup\smash{\underline{\box0}}\\}
\def\|{\verb|}

\usepackage{caption}
\lstset{
  basicstyle={\ttfamily},
  identifierstyle={\small},
  commentstyle={\smallitshape},
  keywordstyle={\small\bfseries},
  ndkeywordstyle={\small},
  stringstyle={\small\ttfamily},
  frame={tb},
  breaklines=true,
  columns=[l]{fullflexible},
  numbers=left,
  xrightmargin=0zw,
  xleftmargin=3zw,
  numberstyle={\scriptsize},
  stepnumber=1,
  numbersep=1zw,
  lineskip=-0.5ex
}
\captionsetup{
    justification=centering % キャプションを中央揃えにする
}

\begin{document}

% 和文表題
\title{様々な状況と環境に対応できる\\PDRベースの屋内位置推定ライブラリの基礎検討}
% 英文表題
\etitle{
	Consideration of a PDR-based indoor location estimation library for various situations and environments}

\affiliate{1}{愛知工業大学大学院 経営情報科学研究科}
\affiliate{2}{愛知工業大学 情報科学部}

% \paffiliate{DICOMO}{マルチメディア,分散,協調とモバイルシンポジウム\\DICOMO2023}

\author{外山 瑠起}{TOYAMA RYUKI}{1}
\author{梶 克彦}{KAZI KATSUHIKO}{2}



\begin{abstract}
	現代社会において,屋内位置推定技術は重要な技術である.屋内の人の動きを把握してビル内のナビゲーションなどに活用するなど様々な活用方法が考えられる.
	多様な状況や環境で屋内位置推定をするには,個々の条件に適した位置推定手法の組み合わせや選択が重要である.
	屋内位置推定の手法としてPDRがある.PDRはスマートフォンなどから得られるセンサデータを元にある地点からの相対的な位置を推定する技術である.
	PDRはスマートフォンなどの機器さえあれば環境に左右されず一定の推定ができる手法である.
	一方でPDRは時間の経過に応じて特有の誤差が蓄積する問題がある.
	そのためPDRの誤差を環境情報などを使用して補正する必要がある.
	本研究の目的は様々な状況と環境に対応できるPDRベースの屋内位置推定ライブラリの開発である.
\end{abstract}

% 表題などの出力
\maketitle

% 本文はここから始まる

\section{はじめに}
屋内位置置推定技術は,現代社会において重要な役割を果たしており様々な活用が期待できる.
屋内位置推定技術が使用される一例として,ショッピングモール施設でのナビゲーションシステムが挙げられる[1].
このシステムでは顧客が店内で商品を探している際,その位置情報を元にしたナビゲーションシステムを通じて目的の商品が置かれている売り場まで案内するシステムである.

屋外における位置推定技術としてGPSが広く利用されているが,屋内環境では建物の壁や天井がGPS衛星からの電波を遮断してしまい,位置推定精度が大きく低下してしまう問題があり,別のアプローチが必要とされている.
屋内位置推定手法として,PDR(Pedestrian Dead Reckoning)が挙げられる.
PDRは主に,加速度計,ジャイロスコープ,磁気センサなどのセンサを利用して歩行者のステップ数,歩行速度,歩行方向を推定する.
その情報を元に歩行者がどのくらいの距離をどの方向に移動したかを累積的に計算して基準となる位置からの相対的な位置を推定する手法である.
他の例として,Wi-Fiの電波を使用した屋内位置推定手法がある.
Wi-Fiを利用した屋内位置推定は,Wi-Fiアクセスポイントからの信号強度を利用して位置推定を行う.特定の地点でのフィンガープリントを予め取得しておきそれと比較して推定を行う手法や,3つのアクセスポイントからの電波強度を利用して三角測量を行う手法がなどがある.

しかしこれらの手法は特定の状況や環境に特化したものであり,すべての屋内環境で同様の効果を発揮するわけではない.例として先ほどあげたWi-Fiを利用した屋内位置推定の場合,地下施設やWi-Fiアクセスポイントが設置が難しい場所では,信号が弱いため正確な位置推定が難しくなる.
PDRの場合各種センサによる誤差が蓄積されるため,時間経過とともに誤差が大きくなってしまう問題がある.そのため長時間の屋内移動場合は精度が低下してしまう.

これらの問題を解決するためには状況や環境に応じて適切な位置推定手法の選択または組みわせる必要がある.
本研究の目的は様々な環境に対応できるオフライン屋内位置推定ライブラリの検討である.
ここでいうオフラインとは,あらかじめ取得したデータを元に位置推定を行うという意味である.
先ほどあげた長時間の屋内移動の場合,PDRで推定した位置情報をWi-Fiの電波強度を利用した位置推定でセンサの誤差を修正すれば,より正解な位置推定を行える.
これらの補正を自由に組み合わせて使えるライブラリの検討を行い,様々な状況に対応できるオフライン屋内位置推定ライブラリを実現する.

\begin{figure}[h]
	\centering
	\includegraphics[width=80mm]{image/first.png}
	\caption{様々な状況と環境に対応できる\\PDRベースの
		屋内位置推定ライブラリの概要}    \label{fig:gt-trajectory}
\end{figure}


\section{関連研究}
絶対位置推定に関する研究がある.
これは特定の基準点からの情報を元に位置を推定する手法である.
例えばBluetoothやWi-Fiなどの電波を利用した手法がある.
屋内に設置した近接特化型のBLEビーコン3つからの電波強度を利用して三角測量を行い位置推定を行う研究\cite{ble-indoor}や
GMMを使用してWi-Fiの電波強度分布をモデル化し,それを元に位置推定を行う研究\cite{wifi-gmm}がある.
これらの手法は設置された機器がない場合や設置が難しい場合は使用できない問題がある.

PDRと絶対位置測位を組み合わせて屋内位置推定を行う手法がある.
PDRとWi-Fiの受信強度を用いたプロキシミティベースの位置推定を行う研究\cite{pdr-wifi}やBLEビーコンの受信信号強度の変異を
利用した移動変異推定とPDRとの併用による累積測位誤差の補正を行う研究\cite{pdr-ble}がある.
またPDRとマップマッチングを組み合わせて位置推定を行う研究\cite{pdr-map}や
歩行時の磁気データのみを用いて位置推定を行う研究\cite{pdr-mag}がある.
これらの研究で示されているようにPDRと絶対位置測位を組み合わせた手法は,
お互いのデメリットを補えるため屋内位置推定をする上で有用な手法である.

本研究でもPDRによる位置推定を行いその結果に対して,BLEビーコンの電波強度を使用した補正やマップマッチングによる
補正をなどが行えるライブラリを検討する.
さらに初期位置や終了位置などの様々な状況の情報で補正できるように
より包括的かつ柔軟な屋内位置推定ライブラリの検討を行う.

\section{PDRベースの屋内位置推定ライブラリの検討}

\subsection{要求仕様}

屋内位置推定は状況や環境によって推定に使用できる情報が異なるため,
ライブラリを作る上でその具体的な例を考える必要がある.
例えば,大学内や病院などのWi-Fiのアクセスポイントが多く設置されている場所では
Wi-Fiの電波強度を利用した位置推定が有効である.
他の例として展示会場や大きなアトリウムなどの広い開放空間が考えられる.
このような場所ではWi-Fiのアクセスポイントの配置が難しく,信号のカバレッジが不均一
になりやすくWi-Fiを利用した位置推定は難しい.
このような場所の場合BLEビーコンを配置してその電波強度を利用した位置推定ができると考えられる.
正しこのような空間で正確な位置推定を行うには,BLEビーコンの多数配置する必要があり,
労力や配置したとしてもメンテナンスコストがかかる可能性がある.
このような場合はPDRを利用した位置推定が有効である.
PDRは歩行者の歩行速度や歩行方向を推定して位置推定を行う手法である.
この手法では歩行者がスマートフォンを持っているだけで良いため,先ほどあげたような
場所でもコストをかけずに位置推定を行えると考えられる.
しかしPDRはセンサのわずかな誤差が,時間経過とともに蓄積され位置推定の誤差が大きくなってしまう問題がある.
そのため長時間の歩行の場合は位置推定の精度が低下してしまう.
このような問題を解決する手法として\cite{pdr-wifi}や\cite{pdr-ble}や方法が提案されている.

これらの手法のように基本的なベースをPDRで位置推定を行い,その結果を補正できるようなライブラリの検討を行う.
ライブラリを作る上での想定環境としてXDR Challenge Track5の環境を元に考える.
XDR Challengeは屋内位置推定の精度を競うコンテストである.
このコンテストのTrack5では歩行者が高速道路のサービスエリアを歩行する.
歩行者は腰にスマートフォンをつけた状態でLiDARと呼ばれる光を使った距離測定技術を搭載した
ハンドヘルドLiDAR(以下,LiDAR)を持ち施設内を歩行する.
参加者にはスマートフォンから得られた,加速度,角速度,磁気センサのデータとLiDARから
得られた正解の歩行軌跡の座標が提供される.
施設内にはBLEビーコンが配置されており,歩行時の各ビーコンからの電波受信強度が提供される.
他に与えられるデータとして,フロアマップ情報,フロアマップにおける各BLEビーコンの配置情報,
歩行者の初期位置,終了位置が与えられる.
本来であれば病院や学校などのより多くの施設で利用できるライブラリにする必要がある.
しかしそれら全てに対応したライブラリを初めから作成するのは難しいためまず基本的な屋内位置推定環境が整っている
XDR Challngeの環境でのライブラリの基礎検討と検証を行う.



\subsection{ライブラリ}
\begin{table*}[ht]
	\centering
	\scalebox{0.65}{
		\begin{tabular}{|c|c|c|c|c|c|c|c|c|c|c|c|c|c|} % Change 'l' to 'c' for center alignment
			\hline
			              & 関数名                                                                 & \multicolumn{4}{c|}{センサ情報}   & \multicolumn{4}{c|}{環境情報}    & \multicolumn{4}{c|}{その他}                                                                                                                                                                                                                                                                                                                          \\ \hline
			              &                                                                     &                              &                              &                              & \multicolumn{1}{c|}{BLEビーコン} &                              & \makecell{磁気                                                                                                } & \multicolumn{2}{c|}{BLEビーコン} & \multicolumn{2}{c|}{正解初期} & \multicolumn{2}{c|}{正解補正}                                                 \\ \cline{6-6} \cline{8-8} \cline{9-9} \cline{10-10}\cline{11-12}\cline{13-14}
			              &                                                                     & 加速度                          & 角速度                          & 角度                           & 電波強度                         & フロアマップ                       & FP                                                                                                            & 基地局位置                        & FP                        & 座標                               & 方向 & 座標                           & 方向 \\ \hline
			基本PDR         & estimate\_trajectory                                                & \multicolumn{1}{c|}{$\circ$} & \multicolumn{1}{c|}{$\circ$} &                              &                              &                              &                                                                                                               &                              &                           & \multicolumn{1}{c|}{$\triangle$} &    &                              &    \\ \hline
			角速度から角度推定     & convert\_to\_angle\_from\_gyro                                      &                              & \multicolumn{1}{c|}{$\circ$} &                              &                              &                              &                                                                                                               &                              &                           &                                  &    &                              &    \\ \hline
			ドリフト補正        & remove\_drift\_in\_angle                                            & \multicolumn{1}{c|}{$\circ$} &                              & \multicolumn{1}{c|}{$\circ$} &                              &                              &                                                                                                               &                              &                           & \multicolumn{1}{c|}{$\circ$}     &    & \multicolumn{1}{c|}{$\circ$} &    \\ \hline
			初期方向補正 フロアマップ & rotate\_trajectory\_to\_optimal\_alignment\_using\_map              & \multicolumn{1}{c|}{$\circ$} &                              & \multicolumn{1}{c|}{$\circ$} &                              & \multicolumn{1}{c|}{$\circ$} &                                                                                                               &                              &                           & \multicolumn{1}{c|}{$\triangle$} &    &                              &    \\ \hline
			初期方向補正 BLE    & rotate\_trajectory\_to\_optimal\_alignment\_using\_ble              & \multicolumn{1}{c|}{$\circ$} & \multicolumn{1}{c|}{$\circ$} &                              & \multicolumn{1}{c|}{$\circ$} &                              &                                                                                                               & \multicolumn{1}{c|}{$\circ$} &                           & \multicolumn{1}{c|}{$\triangle$} &    &                              &    \\ \hline
			マップマッチング補正    & move\_unwalkable\_points\_to\_walkable                              & \multicolumn{1}{c|}{$\circ$} & \multicolumn{1}{c|}{$\circ$} &                              &                              & \multicolumn{1}{c|}{$\circ$} &                                                                                                               &                              &                           &                                  &    &                              &    \\ \hline
			安定歩行区間補正      &                                                                     & \multicolumn{1}{c|}{$\circ$} & \multicolumn{1}{c|}{$\circ$} &                              &                              &                              &                                                                                                               &                              &                           &                                  &    &                              &    \\ \hline
			初期方向補正 BLE FP & rotate\_trajectory\_to\_optimal\_alignment\_using\_ble\_fingerprint
			              & \multicolumn{1}{c|}{$\circ$}                                        & \multicolumn{1}{c|}{$\circ$} &                              &                              &                              &                              &                                                                                                               & \multicolumn{1}{c|}{$\circ$} &                           &                                  &    &                                   \\ \hline
		\end{tabular}
	}
	\caption{関数に必要な情報とその対応表} \label{}
	\textit{注: $\circ$は必須引数,$\triangle$はオプショナル引数を示す} \label{tab:my_label}
\end{table*}


\subsection{ライブラリ}

\begin{table*}[ht]
	\centering
	\scalebox{0.65}{
		\begin{tabular}{|c|c|c|c|c|c|c|c|c|c|c|c|c|c|} % Change 'l' to 'c' for center alignment
			\hline
			                & 関数名                                                                 & \multicolumn{4}{c|}{センサ情報}   & \multicolumn{4}{c|}{環境情報}    & \multicolumn{4}{c|}{その他}                                                                                                                                                                                                                                                                                                                          \\ \hline
			                &                                                                     &                              &                              &                              & \multicolumn{1}{c|}{BLEビーコン} &                              & \makecell{磁気                                                                                                } & \multicolumn{2}{c|}{BLEビーコン} & \multicolumn{2}{c|}{正解初期} & \multicolumn{2}{c|}{正解補正}                                                 \\ \cline{6-6} \cline{8-8} \cline{9-9} \cline{10-10}\cline{11-12}\cline{13-14}
			                &                                                                     & 加速度                          & 角速度                          & 角度                           & 電波強度・AP情報                        & フロアマップ                       & FP                                                                                                            & 基地局位置                        & FP                        & 座標                               & 方向 & 座標                           & 方向 \\ \hline
			基本PDR           & estimate\_trajectory                                                & \multicolumn{1}{c|}{$\circ$} & \multicolumn{1}{c|}{$\circ$} &                              &                              &                              &                                                                                                               &                              &                           & \multicolumn{1}{c|}{$\triangle$} &    &                              &    \\ \hline
			角速度から角度推定       & convert\_to\_angle\_from\_gyro                                      &                              & \multicolumn{1}{c|}{$\circ$} &                              &                              &                              &                                                                                                               &                              &                           &                                  &    &                              &    \\ \hline
			ドリフト補正          & remove\_drift\_in\_angle                                            & \multicolumn{1}{c|}{$\circ$} &                              & \multicolumn{1}{c|}{$\circ$} &                              &                              &                                                                                                               &                              &                           & \multicolumn{1}{c|}{$\circ$}     &    & \multicolumn{1}{c|}{$\circ$} &    \\ \hline
			初期進行方向補正 フロアマップ & rotate\_trajectory\_to\_optimal\_alignment\_using\_map              & \multicolumn{1}{c|}{$\circ$} &                              & \multicolumn{1}{c|}{$\circ$} &                              & \multicolumn{1}{c|}{$\circ$} &                                                                                                               &                              &                           & \multicolumn{1}{c|}{$\triangle$} &    &                              &    \\ \hline
			初期進行方向補正 BLE    & rotate\_trajectory\_to\_optimal\_alignment\_using\_ble              & \multicolumn{1}{c|}{$\circ$} & \multicolumn{1}{c|}{$\circ$} &                              & \multicolumn{1}{c|}{$\circ$} &                              &                                                                                                               & \multicolumn{1}{c|}{$\circ$} &                           & \multicolumn{1}{c|}{$\triangle$} &    &                              &    \\ \hline
			マップマッチング補正      & move\_unwalkable\_points\_to\_walkable                              & \multicolumn{1}{c|}{$\circ$} & \multicolumn{1}{c|}{$\circ$} &                              &                              & \multicolumn{1}{c|}{$\circ$} &                                                                                                               &                              &                           &  \multicolumn{1}{c|}{$\triangle$}                                 &    &                              &    \\ \hline
			初期進行方向補正 BLE FP & rotate\_trajectory\_to\_optimal\_alignment\_using\_ble\_fingerprint
			                & \multicolumn{1}{c|}{$\circ$}                                        & \multicolumn{1}{c|}{$\circ$} &                              &                              &                              &                              &                                                                                                               & \multicolumn{1}{c|}{$\circ$} &  \multicolumn{1}{c|}{$\triangle$}                                                          &  &    &                                   \\ \hline
		\end{tabular}
	}
	\caption{関数に必要な情報とその対応表} \label{}
	\textit{注: $\circ$は必須引数,$\triangle$はオプショナル引数を示す} \label{tab:my_label}
\end{table*}


関数に必要な引数の情報とその対応表を表1に示す.
詳しい関数の説明や内部実装については後述する.
引数の情報は大きく分けてセンサ情報,環境情報,その他の3つに分けられる.
センサ情報はスマートフォンから得られる加速度,角速度,BLEビーコンの電波情報などが含まれる.
環境情報はフロアマップ情報,フロアマップにおける各BLEビーコンの配置情報などが含まれる.
これらの環境情報は全てセンサデータが与えられる前に得られる情報である.
その他はセンシング中,またはセンシング前に得られる情報であり,初期位置,終了位置などの情報が該当する.


\begin{table}[ht]
	\centering
	\begin{tabular}{lll}
		\toprule
		カラム名 & 単位        & データ型  \\
		\midrule
		ts   & s (秒)     & float \\
		x    & m/s\(^2\) & float \\
		y    & m/s\(^2\) & float \\
		z    & m/s\(^2\) & float \\
		\bottomrule
	\end{tabular}
	\caption{加速度 DF}
\end{table}

\begin{table}[ht]
	\centering
	\begin{tabular}{lll}
		\toprule
		カラム名 & 単位             & データ型  \\
		\midrule
		ts   & s (秒)          & float \\
		x    & rad/s (ラジアン/秒) & float \\
		y    & rad/s (ラジアン/秒) & float \\
		z    & rad/s (ラジアン/秒) & float \\
		\bottomrule
	\end{tabular}
	\caption{角速度 DF}
\end{table}


\begin{table}[ht]
	\centering
	\label{tab:first-coord-dict}
	\begin{tabular}{lll}
		\hline
		      & {データ型}         & {説明}          \\ \hline
		key   & \texttt{str}   & xまたはy         \\ \hline
		value & \texttt{float} & \makecell{座標} \\ \hline
	\end{tabular}
	\caption{正解初期座標 DICT}
\end{table}


\begin{table}[ht]
	\centering
	\begin{tabular}{lll}
		\toprule
		カラム名 & 単位      & データ型  \\
		\midrule
		ts   & s (秒)   & float \\
		x    & m(メートル) & float \\
		y    & m(メートル) & float \\
		\bottomrule
	\end{tabular}
	\caption{座標DF}
\end{table}


\begin{table}[ht]
	\centering
	\begin{tabular}{lll}
		\toprule
		カラム名 & 単位         & データ型  \\
		\midrule
		ts   & s (秒)      & float \\
		x    & rad (ラジアン) & float \\
		y    & rad (ラジアン) & float \\
		z    & rad (ラジアン) & float \\
		\bottomrule
	\end{tabular}
	\caption{角度 DF}
\end{table}

説明する関数の引数に必要な情報は全て表1と紐づいている.
言語にはPythonを使用した.Pythonはデータ解析や機械学習などの分野で広く使われており,
ライブラリを使用するユーザーにとっても比較的扱いやすい利点がある.
まず基本的なPDRの処理を行う関数をListing\ref{lst:pdr-trajectory}に示す.
この関数では加速度データフレーム(以下,DF),角速度DFを使用して位置推定を行う.
加速度DF,角速度DFのデータフレームのカラム名とデータ型を表2,表3に示す.
オプショナル引数として正解初期座標(ground\_truth\_first\_point)を与えられる.
正解初期座標は辞書型で表4に示す.
戻り値は時間経過に伴う2次元座標のDFと角度のDFであり,それぞれのカラム名とデータ型を表5,表6に示す.
戻り値で角度DFを返す理由として,次の補正処理をする際に角速度よりも扱いやすいためである.

歩幅の推定を行っている研究は多くある.
機会学習を用いた研究\cite{stride-length-auto-learning},
多変量解析を用いた研究\cite{stride-length-multi},
超音波センサーガジェットを用いた研究\cite{stride-length-ultrasonic}などがある.
本関数の内部処理では歩幅の値は固定値として扱っている.
本来であれば歩幅は身長,性別,年齢などの複数の要素によって動的に変化するため
固定値なのはありえず,先ほど挙げた研究のように歩幅を推定する必要がある.
しかし本ライブラリの目的は正確な歩幅を用いたPDRによる位置推定ではない.
PDRで推定した歩行軌跡を環境情報などを用いて補正を行い軌跡全体の最適化を行えるライブラリの検討である.
そのため歩幅の推定は行わず固定値として扱う.
また同様の理由で歩行タイミングの検出も正確には行わず,加速度の値が特定の閾値を超えた時に
歩行タイミングとして扱っている.
図\ref{fig:pdr}にリスト\ref{lst:pdr-trajectory}を用いてPDRによる位置推定を行った結果を示す.
LiDARで取得した座標をもとに出力された軌跡を図\ref{fig:gt-trajectory}に示す.
これを本論では正解軌跡とする.
図\ref{fig:pdr}と図\ref*{fig:gt-trajectory}を比較するとPDRによる軌跡は正解軌跡と比べて大きくずれているのがわかる.
PDR特有の解決すべきものとして
軌跡そのものの形状を正解奇跡に近づける問題と絶対位置との関連付けの問題がある.
本ライブラリを用いてこれらの問題を解消し正解軌跡に近づけていく.

% 文書内
\begin{lstlisting}[caption={基本PDR}, label=lst:pdr-trajectory]
Axis2D = Literal["x", "y"]
def estimate_trajectory(
    acc_df: pd.DataFrame,
    gyro_df: pd.DataFrame,
    *,
    ground_truth_first_point: dict[Axis2D, float] | None = None,
) -> tuple[pd.DataFrame, pd.DataFrame]:
\end{lstlisting}

\begin{figure}[ht]
	\centering
	\includegraphics[width=80mm]{image/pdr.jpg}
	\caption{基本PDRの軌跡}    \label{fig:pdr}
\end{figure}

Listing\ref{lst:pdr-trajectory}に示される関数に正解初期座標を
を与えたのが図\ref{fig:pdr-move}である.
予め正解座標が判明している場合はPDRによる軌跡の初期位置を補正できる.

\begin{figure}[ht]
	\centering
	\includegraphics[width=80mm]{image/gt2.jpg}
	\caption{正解軌跡}    \label{fig:gt-trajectory}
\end{figure}

\begin{figure}[ht]
	\centering
	\includegraphics[width=80mm]{image/pdr-move.jpg}
	\caption{正解初期座標が存在}    \label{fig:pdr-move}
\end{figure}

\input{./3/remove-drift.tex}
\input{./3/rotate-trajectory-map.tex}

フロアマップ情報を用いた初期方向補正ではマップの存在可能な点の分布によっては正しく機能しない場合がある.
別の方法としてBLEビーコンの基地局の位置情報を用いた初期方向補正を行う関数を提供する.
関数をListing\ref{lst:rotate-trajectory-using-ble}に示す.
この関数では加速度DF,角度DF,BLEビーコンの受信電波DF, BLEビーコンの基地局DFを受け取る.
BLEビーコンの受信電波DFとBLEビーコンの基地局DFのカラム名とデータ型を表6,表7に示す.
戻り値は時間経過に伴う2次元座標のDFと角度のDFを返す.
一定の強いRSSIの電波を受信した際の時間情報を元に時間的に近い推定軌跡の座標を取得する.
図\ref{fig:ble-merge}に示した図は時間的に近い推定軌跡の座標を時間経過に応じた色で表しており
青色の座標が配置されたBLEビーコンの座標を表している.

推定した軌跡の受信したBLEビーコンの基地局の座標との距離を計算する.
この総和が最小となるような回転角度をグリッドサーチで探し最適な角度に補正を行う.
BLEビーコンの基地局の座標との距離を計算する.

\begin{lstlisting}[caption={BLEビーコンの基地局の位置情報を使用した初期方向補正}, label=lst:rotate-trajectory-using-ble]
def rotate_trajectory_to_optimal
		_alignment_using_ble(
    acc_df: pd.DataFrame,
    angle_df: pd.DataFrame,
    ble_scans_df: pd.DataFrame,
    ble_position_df: pd.DataFrame,
    *,
    ground_truth_first_point: dict[Axis2D, float] | None = None,
) -> tuple[pd.DataFrame, pd.DataFrame]:
\end{lstlisting}


\begin{figure}[ht]
	\centering
	\includegraphics[width=80mm]{image/ble-merge.jpg}
	\caption{強いビーコン電波を受信した際の\\時間的に近い軌跡の座標}    \label{fig:ble-merge}
\end{figure}

\begin{table}[ht]
	\centering
	\begin{tabular}{lll}
		\toprule
		カラム名      & 単位    & データ型  \\
		\midrule
		ts        & s (秒) & float \\
		bdaddress & なし    & str   \\
		rssi      & dBm   & int   \\
		\bottomrule
	\end{tabular}
	\caption{BLEビーコン受信電波 DF}
\end{table}

\begin{table}[ht]
	\centering
	\begin{tabular}{lll}
		\toprule
		カラム名        & 単位 & データ型  \\
		\midrule
		bdaddress   & なし & str   \\
		x           & m  & float \\
		y           & m  & float \\
		floor\_name & なし & str   \\
		\bottomrule
	\end{tabular}
	\caption{BLEビーコン基地局 DF}
\end{table}



\begin{table}[ht]
	\centering
	\begin{tabular}{lll}
		\toprule
		カラム名        & 単位 & データ型  \\
		\midrule
		bdaddress   & なし & str   \\
		x           & m  & float \\
		y           & m  & float \\
		floor\_name & なし & str   \\
		\bottomrule
	\end{tabular}
	\caption{BLEビーコン基地局 DF}
\end{table}


\begin{table}[ht]
	\centering
	\begin{tabular}{lll}
		\toprule
		カラム名        & 単位      & データ型  \\
		\midrule
		ts          & s (秒)   & float \\
		x           & m(メートル) & float \\
		y           & m(メートル) & float \\
		z           & m(メートル) & float \\
		bdaddress   & なし      & str   \\
		rssi        & dBm     & int   \\
		floor\_name & なし      & str   \\
		\bottomrule
	\end{tabular}
	\caption{BLEビーコンFPのDF}
\end{table}


\begin{figure}[ht]
	\centering
	\includegraphics[width=80mm]{image/fingerprint-rotate.jpg}
	\caption{回転後の軌跡}    \label{fig:fingerprint-rotate}
\end{figure}


\begin{lstlisting}[caption={BLEビーコンのFPを使用した初期方向補正}, label=lst:rotate-trajectory-using-ble-fingerprint]
def rotate_trajectory_to_optimal
          _alignment_using_ble_fingerprint(
    acc_df: pd.DataFrame,
    angle_df: pd.DataFrame,
    ble_scans_df: pd.DataFrame,
    ble_fingerprint_df: pd.DataFrame,
    floor_name: str,
    *,
    ground_truth_first_point: dict[Axis2D, float] | None = None,
) -> tuple[pd.DataFrame, pd.DataFrame]:
\end{lstlisting}
\input{./3/map-matching.tex}





\section{ライブラリの検証と他環境における検討}
この章では提案したライブラリの有効性を検証する.
検証はxDR Challenge 2023での評価を基に行い,
また他環境でこれらのライブラリが適用可能かを検討する.

\subsection{xDR Challenge 2023環境での評価}

 \begin{table*}[ht]
    \centering
    \caption{提供データの概要}
    \scalebox{0.8}{
    \begin{tabular}{|l|l|l|l|l|}
        \hline
        データタイプ & 測定デバイス & レート & 訓練データ & スコアリングデータ \\ \hline
        加速度 & AQUOS Sense 6 & 約100Hz & 使用可能 & 使用可能  \\ \hline
        角速度 & AQUOS Sense 6 & 約100Hz & 使用可能 & 使用可能 \\ \hline
        磁気 & AQUOS Sense 6 & 約100Hz & 使用可能 & 使用可能 \\ \hline
        BLE RSSI & AQUOS Sense 6 & 10Hzのビーコンから送信、AQUOS Sense 6で受信時に記録 & 使用可能 & 使用可能 \\ \hline
        正解位置 (Ground Truth) (x, y, z) & ZEB-Horizon & 約100Hz & 使用可能 & 始めと終わりのみ使用可能\\ \hline
        正解姿勢 (Ground Truth) (四元数) & ZEB-Horizon & 約100Hz & 使用可能 & 始めと終わりのみ使用可能\\ \hline
        正解階層名 (Ground Truth) & - & 各パスの1階層名 & 使用可能 & 使用可能 \\ \hline
    \end{tabular}
    \label{table:data}
  }
\end{table*}



xDR Challenge 2023ではPDRベンチマーク標準化委員会によって
提供された評価フレームワークを使用して評価が行われた.
このフレームワークではl\_ce(CE:円形誤差),l\_ca(CA\_l:局所空間における円形精度),
l\_eag(誤差蓄積勾配),l\_ve(VE:速度誤差),l\_obstacle(障害物回避要件)
の5つの評価指標が用いられた.
総合評価指標は式\ref{eq:evaluation_index}に示す式で計算される.
このライブラリを使って得られた評価と各指標の重みを表\ref{table:evaluation_index}に示す.
I\_ce,I\_eag, I\_ve,I\_obstacleでは一定の精度を得られた.
しかしI\_caでは精度が低かった.
I\_caの値が低いと場合局所空間における位置推定誤差の分布が広い.
これは環境条件の変化やセンサデータの微妙な違いが,位置推定結果に大きな影響を与える可能性が高い.
この問題を解決するためには,PDRアルゴリズムを改善し,より精度の高い位置推定を行う必要がある.

\begin{equation}
	\begin{aligned}
		I_i = & W_{ce} \times I_{ce} + W_{ca} \times I_{ca}                                        \\
		      & + W_{eag} \times I_{eag} + W_{ve} \times I_{ve} + W_{obstacle} \times I_{obstacle}
	\end{aligned}
	\label{eq:evaluation_index}
\end{equation}


\begin{table}[ht]
  \caption{評価指数の概要}
	\centering
	\begin{tabular}{l|l|l}
		\hline
		指標                        & 値 (\%) & 重み   \\ \hline
		I\_ce(CE:円形誤差)            & 88.55  & 0.25 \\
		I\_ca(CA\_l:局所空間における円形精度) & 62.51  & 0.20 \\
		I\_eag(EAG:誤差蓄積勾配)        & 93.02  & 0.25 \\
		I\_ve(VE:速度誤差)            & 95.55  & 0.15 \\
		I\_obstacle(障害物回避要件)      & 93.48  & 0.15 \\
		I (総合評価指数)                & 86.25  &      \\ \hline
	\end{tabular}
	\label{table:evaluation_index}
\end{table}


\subsection{駅の構内での検討}
駅構内での位置推定をする場合を考える.
駅の改札は地上から続いてるものもあれば地下にあるものもある.
地下の場合は特に衛星からの電波が届きにくい場所であるためGPSが有効ではない.
このような環境ではPDRが有効な手法である.
駅の改札の位置は工事などがない限り,基本的に固定位置から変化はしない.
改札を通った時の位置をPDRの開始地点を正解の初期座標として使用できる.
改札を通って出た後,乗り換えを行う場合がある.
このような場合は次の改札口を正解補正座標として利用できる.
ICなどを使って駅改札を通った場合,ユーザを一意に識別できる.
そのため特定のユーザが乗り換えをしたという情報を収集するのは比較的容易である.
正解初期地点と正解補正座標を利用すればListing\ref{lst:remove-drift}に示したドリフト補正を適用できる.
また全ての駅ではないがある一定規模以上の駅構内の場合フロアマップ情報が入手できる可能性が高い.
その場合フロアマップ情報を用いたListing\ref{lst:map-matching}のマップマッチング補正が適用できる.

\subsection{大学のキャンパスでの検討}
大学のキャンパスで位置推定をする場合を考える.
大学には屋外環境と屋内環境がある.
建物間の移動経路を把握する場合はGPSが有効である.
しかし大学の建物内での移動経路を把握する場合GPSでは困難である.
このような場合にPDRを軸とした移動経路の把握を検討する.
大学というのは研究室やサ―クルなど異なるコミュニティが混在している.
それらは1つの組織が大本で管理しているのではなく個々が独立運営している.
このような場所でBLEビーコンを配置する場合,各コミュニティへの申請のコストや
,場合によっては配置を拒否される可能性がある.
BLEビ―コン以外の電波の利用を考えた場合,Wi-Fiの電波の利用が検討できる.
Wi-Fiの基地局なら基本的にどのコミュニティにも配置がしてあり,設置コストの面でBLEビーコンと比べると低い.
しかし既知のWi-Fiの基地局位置情報の把握はコストが大きい.
そのためこのような場所ではWi-Fiの電波を使ったFP補正が有効だと考えられる.
FPを使った手法なら基地局の位置情報を把握していない場合でも利用できる.
3章ではBLEビ―コンの元にFP処理を行う関数を実装した.
Wi-FiとBLEは通信範囲や消費電力などで異なる点はあり,
内部の処理や閾値を変化させる必要はあるが,基本的に与える引数やそのデ―タ形式を揃えれば同様に適用できる.
また部屋に出入りする際には固定の位置の出入り口がある.
個人がそこを出入りしたという情報を取得できれば,
正解初期座標や正解補正座標としてドリフト除去を適用できる.



\section{まとめ}
本論文では,環境情報などを利用したPDRベースの位置推定ライブラリの基礎検討を行った.
補正に利用できる情報をセンサ情報,環境情報,その他の3つに分類し,それぞれの情報を用いた補正処理を提案した.
その結果として,xDR Challenge 環境下では一定の精度を獲得した.
また他環境においても本ライブラリが適用可能であるか検討を行った.
課題としては,PDRアルゴリズムの改善や挙げられる.
また本論文は2次元の屋内位置推定のみを想定したライブラリ構成となっている.
現実の屋内では3次元に構成されるものが多いため,本ライブラリを3次元空間に適用できるような拡張を検討したい.
具体的にはスマートフォンの気圧センサの値を使用すれば相対的な階層間の移動の検知が可能である.
これとフロアマップ情報を組み合わせることで3次元空間での位置推定が実現できると考えられる.










\bibliography{dicomo}
\bibliographystyle{junsrt}

\end{document}
