
\section{PDRベースの屋内位置推定ライブラリの検討}

\subsection{要求仕様}
屋内位置推定は状況や環境によって推定に使用できる情報が異なるため,
ライブラリを作る上でその具体的な例を考える必要がある.
例えば,大学内や病院などのWi-Fiのアクセスポイントが多く設置されている場所では
Wi-Fiの電波強度を利用した位置推定が有効である.
他の例として展示会場や大きなアトリウムなどの広い開放空間が考えられる.
このような場所ではWi-Fiのアクセスポイントの配置が難しく,信号のカバレッジが不均一
になりやすくWi-Fiを利用した位置推定は難しい.
このような場所の場合BLEビーコンを配置してその電波強度を利用した位置推定ができると考えられる.
正しこのような空間で正確な位置推定を行うには,BLEビーコンの多数配置する必要があり,
労力や配置したとしてもメンテナンスコストがかかる可能性がある.
このような場合はPDRを利用した位置推定が有効である.
PDRは歩行者の歩行速度や歩行方向を推定して位置推定を行う手法である.
この手法では歩行者がスマートフォンを持っているだけで良いため,先ほどあげたような
場所でもコストをかけずに位置推定を行えると考えられる.
しかしPDRはセンサのわずかな誤差が,時間経過とともに蓄積され位置推定の誤差が大きくなってしまう問題がある.
そのため長時間の歩行の場合は位置推定の精度が低下してしまう.
このような問題を解決する手法として\cite{pdr-wifi}や\cite{pdr-ble}や方法が提案されている.

これらの手法のように基本的なベースをPDRで位置推定を行い,その結果を補正できるようなライブラリの検討を行う.
ライブラリを作る上での想定環境としてXDR Challenge Track5の環境を元に考える.
XDR Challengeは屋内位置推定の精度を競うコンテストである.
このコンテストのTrack5では歩行者が高速道路のサービスエリアを歩行する.
歩行者は腰にスマートフォンをつけた状態でLiDARと呼ばれる光を使った距離測定技術を搭載した
ハンドヘルドLiDAR(以下,LiDAR)を持ち施設内を歩行する.
参加者にはスマートフォンから得られた,加速度,角速度,磁気センサのデータとLiDARから
得られた正解の歩行軌跡の座標が提供される.
施設内にはBLEビーコンが配置されており,歩行時の各ビーコンからの電波受信強度が提供される.
他に与えられるデータとして,フロアマップ情報,フロアマップにおける各BLEビーコンの配置情報,
歩行者の初期位置,終了位置が与えられる.
本来であれば病院や学校などのより多くの施設で利用できるライブラリにする必要がある.
しかしそれら全てに対応したライブラリを初めから作成するのは難しいためまず基本的な屋内位置推定環境が整っている
XDR Challngeの環境でのライブラリの基礎検討と検証を行う.
