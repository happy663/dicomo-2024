

\subsection{要求仕様}
PDRと他の情報を使ってライブラリを作成する上で,
どのような状況や環境が存在し補正に利用できるのかその具体的な例を考える必要がある.
例えば大学内や病院などのWi-Fiのアクセスポイントが多く設置されている場所では,
Wi-Fiの電波強度を利用した位置推定が有効である.
他の例として展示会場や大きなアトリウムなどの広い開放空間が考えられる.
このような場所ではWi-Fiのアクセスポイントの配置が難しく,
信号のカバレッジが不均一になりやすくWi-Fiを利用した位置推定は難しい.
このような場所の場合BLEビーコンを配置してその電波強度を利用した位置推定が有効である.
また2章で示したように\cite{pdr-wifi}\cite{pdr-ble}などのPDRと電波を利用した推定に関する研究は盛んに行われている.
このように電波を使った手法は多くの場所で有効であり,補正に利用可能な情報として重要度が高い.
そのため本ライブラリにおいても採用を行う.
他に補正に利用可能な情報としてフロアマップ情報がある.
フロアマップ情報は多くの場所で比較的入手が容易だと思われる.
そのため本ライブラリにおいても採用を行う.
磁気やカメラなどの情報は,磁気データはデータが繊細であり電波と比べると補正に利用する難易度が高い,
カメラはプライバシーの問題などの問題があり本ライブラリの基礎段階において採用しない.
また気圧センサは基礎段階として3次元空間を推定対象としないため採用しない.

本ライブラリの検討および検証においてxDR Challenge 2023の環境を利用した.
xDR Challnge 2023はPDRベンチマーク委員会が主催する屋内位置推定の精度を競うコンテストである.
このコンテストの特定トラックでは人が高速道路のサービスエリアを歩行する.
歩行者は腰にスマートフォンをつけた状態でLiDARと呼ばれる光を使った距離測定技術を搭載したハンドヘルドLiDAR(以下,LiDAR)を持ち施設内を歩く.
コンテストの参加者には,歩行時のデータが複数のトレーニングデータとして提供される.
スマートフォンから加速度,角速度,磁気センサのデータ,および施設内に配置された各BLEビーコンのAP情報と受信電波強度が提供される.
LiDARからは、正解歩行軌跡のデータが提供される。
またフロアマップ情報,フロマップ上での各BLEビーコンの基地局情報,歩行者の初期位置,終了位置が与えられる.
コンテスト本番ではLiDARからの正解の移動軌跡が提供されず,それ以外の情報を使って歩行者の移動経路の推定を行う.
xDR Challenge 2023の環境は,本ライブラリの要求仕様に適合しており,
ライブラリの有効性を検証するには適している.よってこの環境を基に検討および検証を行う.
