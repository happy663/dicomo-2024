

\section{関連研究}
絶対位置推定に関する研究がある.
これは特定の基準点からの情報を元に位置を推定する手法である.
例えばBluetoothやWi-Fiなどの電波を利用した推定手法がある.
これらの電波を利用した推定手法はTriangulation方式,Fingerprint(以下,FP)方式,Proximity方式の3つに分類される
\cite{wireless-lan-summary}.
Triangulation方式を使用した研究として屋内に設置した近接特化型のBLEビーコン3つからの電波強度を利用して三角測量を行い位置推定を行う研究がある.\cite{ble-indoor}\cite{ble-tandem}\cite{triangulation-kalman}
FP方式は特定の地点でのAPからの電波強度モデルを作成して,
実際の測定値をこのモデルの情報と照合して位置を推定する方式である.
この方式はデータ収集コストが大きい点やモデルを作成しても
環境の変化によってモデルの信頼性が低下してしまう問題があり,
それらの問題への対策を行った様々な研究がある
\cite{gaussian-mixture-model}
\cite{wireless-lan-cost-reduction}
\cite{fingerprint-auto-update}
\cite{wi-fi-fingerprint-domain}.
Promixity方式は特定のAPからの強い電波を受信した際,
そのAP付近にいると見なし推定する手法である.
これらの手法は状況に合わせて使いわけを行い組み合わせることでより精度の高い位置推定ができる.
Proximity方式とFP方式を併用した推定手法に関する研究がある\cite{proximity-fingerprint}.
この研究ではWi-FiのFP方式で位置を推定する前に有用に推定できるAPの絞りこみを行っている.
電波以外推定手法として磁気データを使用して位置推定を行う研究\cite{pdr-mag}や,
赤外線を使用して位置推定を行う研究\cite{infrared},カメラを利用した研究\cite{camera}などがある.

PDRと絶対位置推定を組み合わせて屋内位置推定を行う研究がある.
PDRは手法の性質上ある地点からの相対的な位置を推定する手法であるため,
位置推定をするには絶対位置推定と組み合わせる必要がある.
またPDRには1章で述べた誤差が蓄積する問題も存在する.
PDRとWi-Fiの受信強度を用いたProximityの位置推定を行う研究\cite{pdr-wifi}がある.
この研究では加速度センサ,地磁気センサ,気圧センサの値を用いて既知地点からの位置推定を行う.
Wi-Fi APからの電波受信強度が特定の閾値を越えている場合はProximityの位置推定に切り替える.
これによってPDRで生じる誤差の補正を行っている.
BLEビーコンの受信信号強度の変異を利用した移動変異推定とPDRとの併用による累積測位誤差の補正を行う研究\cite{pdr-ble}がある.
この手法ではWi-FiのAPからの情報を使うのでなくBLEビーコンを配置してその電波と位置情報をもとにPDRの誤差の補正を行っている.
またPDRとマップマッチングを組み合わせて位置推定を行う研究\cite{pdr-map}がある.
この手法ではマップマッチングを用いて歩幅を動的に更新を行い誤差を低減したPDR手法を提案している.
これらの研究で示されているようにPDRと絶対位置測位を組み合わせた手法は,
お互いのデメリットを補えるため屋内位置推定をする上で有用な手法である.

本研究でもPDRによる位置推定を行い,その結果に対してBLEビーコンの電波強度を使用した補正やマップマッチングによる
補正をなどが行えるライブラリを検討する.
さらに初期位置や終了位置の有無などの様々な状況の情報で補正できるようなライブラリの検討を目指す.
