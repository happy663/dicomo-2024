
\section{ライブラリの検証と他環境における検討}
この章では,提案したライブラリの有効性を検証する.
検証はXDR Challengeでの評価を元に行い.
また他環境でこれらのライブラリが適用可能かを検討する.

\subsection{XDR Challenge環境での評価}
XDR ChellengeではPDRベンチマーク標準化委員会によって
提供された評価フレームワークを使用して評価が行われた.
このフレームワークではl\_ce(CE:円形誤差),l\_ca(CA\_l:局所空間における円形精度),
l\_eag(誤差蓄積勾配),l\_ve(VE:速度誤差),l\_obstacle(障害物回避要件)
の5つの評価指標が用いられた.
総合評価指標は式\ref{eq:evaluation_index}に示す式で計算される.
このライブラリを使って得られた評価と各指標の重みを表\ref{table:evaluation_index}に示す.
I\_ce,I\_eag, I\_ve,I\_obstacleでは一定の精度を得られた.
しかしI\_caでは精度が低かった.
I\_caの値が低いと場合局所空間における位置推定誤差の分布が広い.
これは環境条件の変化やセンサデータの微妙な違いが,位置推定結果に大きな影響を与える可能性が高い.
この問題を解決するためには,PDRアルゴリズムを改善し,より精度の高い位置推定を行う必要がある.

\begin{equation}
	\begin{aligned}
		I_i = & W_{ce} \times I_{ce} + W_{ca} \times I_{ca}                                        \\
		      & + W_{eag} \times I_{eag} + W_{ve} \times I_{ve} + W_{obstacle} \times I_{obstacle}
	\end{aligned}
	\label{eq:evaluation_index}
\end{equation}

\begin{table}[ht]
	\centering
	\begin{tabular}{l|l|l}
		\hline
		指標                        & 値 (\%) & 重み   \\ \hline
		I\_ce(CE:円形誤差)            & 88.55  & 0.25 \\
		I\_ca(CA\_l:局所空間における円形精度) & 62.51  & 0.20 \\
		I\_eag(EAG:誤差蓄積勾配)        & 93.02  & 0.25 \\
		I\_ve(VE:速度誤差)            & 95.55  & 0.15 \\
		I\_obstacle(障害物回避要件)      & 93.48  & 0.15 \\
		I (総合評価指数)                & 86.25  &      \\ \hline
	\end{tabular}
	\caption{評価指数の概要}
	\label{table:evaluation_index}
\end{table}




\subsection{駅の構内での検討}
駅構内でのPDRでの位置推定をする場合を考える.

駅の場合,改札口という通過するポイントがある.
通過した情報を仮に利用できるとしたら,本ライブラリで提案した補正座標を用いたドリフト補正処理を使えると考えられる.
また大規模な駅構内の場合,フロアマップ情報が公開されている可能性も高い.
そのためフロアマップ情報を用いた初期方向補正やマップマッチング補正も有効だと考えられる.


\subsection{大学のキャンパスでの検討}
大学のキャンパスでのPDRでの位置推定をする場合を考える.
大学のキャンパスの場合Wi-Fiのアクセスポイントが多く設置されている.
本ライブラリではBLEビーコンを利用した補正を行っているがBLEをWi-Fiに置き換えれば同様の処理が可能である.
設置する形式のBLEビーコンと違いあらかじめ存在するWi-Fiの基地局情報を把握するのはコストがかかる可能性がある.
このような場合にライブラリが提供するFPを使った補正が有効だと考えられる.
